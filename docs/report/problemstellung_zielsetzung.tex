\section{Problemstellung und Zielsetzung}

Im Rahmen des Machine Learning Praktikums beschäftigen wir uns mit der folgenden Problemstellung: unser Ziel ist es, einen Klassifikator zu entwickeln, der Bilder von Hautläsionen in maligne und benigne Läsionen unterteilen kann. Maligne Hautläsionen sind die Läsionen, die für den Menschen gefährlich bis sogar tödlich werden könne, während benigne Hautläsionen die gutartigen Läsionen darstellen. Unser Projekt basiert dabei auf der Arbeit von \citet{esteva2017dermatologist}, wobei wir die originale Problemstellung jedoch etwas abgewandelt haben. Während \citet{esteva2017dermatologist} viele verschiedene Arten von Hautläsionen unterschieden haben, wollen wir lediglich zwischen zwei Klassen, nämlich den gutartigen und den bösartigen Läsionen, unterscheiden, was die Problemstellung etwas vereinfacht.\\
Unser Ziel war es eine möglichst hohe Genauigkeit zu erreichen und vor allem die Anzahl der falsch Negativen möglichst gering zu halten. Denn im Zweifel soll der Klassifikator eine Läsion als maligne klassifizieren, auch wenn sie eigentlich benigne ist, anstatt eine maligne Läsion, die tödlich verlaufen könnte, zu verharmlosen und als benigne zu klassifizieren. Im Folgenden werden wir genauer auf die Methodik eingehen, die hinter unserem Klassifikator steckt und welche Ergebnisse dieser auf unbekannte Bilder von Hautläsionen liefert. 