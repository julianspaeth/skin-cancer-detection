\documentclass[a4paper, doc]{apa6}
\usepackage[utf8]{inputenc}
\usepackage[ngerman]{babel}
\usepackage[T1]{fontenc}
\usepackage{setspace}
\onehalfspacing
\usepackage{amsmath}
\usepackage{amsfonts}
\usepackage{amssymb}
\usepackage{graphicx}
\usepackage{listings}
\usepackage{calc}
\usepackage{natbib}
\usepackage{lipsum}
\usepackage{todonotes}



\title{Projektbericht: Skin Cancer Detection}
\shorttitle{Skin Cancer Detection}
\author{Florence Lopez, Jonas Einig, Julian Späth }
\affiliation{Department of Computer Science, University of Tübingen}


\begin{document}

\maketitle
\begin{abstract}

In der Medizin unterscheidet man bei Hautverletzungen oder Hautveränderungen zwischen benignen und malignen Läsionen. Die malignen Läsionen gelten dabei als die bösartigen Hautveränderungen, die auf Hautkrebs hindeuten. Wir entwickelten daher auf der Grundlage einer anderen Arbeit einen Klassifikator, der aus einem Datensatz von Hautläsionen zwischen Benignen und Malignen unterscheidet. Durch diesen Klassifikator soll es zukünftig einfacher sein, bösartige Hautläsionen schneller zu erkennen, um diese effizient zu behandeln. Daher arbeiteten wir zusätzlich noch an einer mobilen Anwendung, welche es Nutzern ermöglicht, ihre Haut vorerst ohne eine ärztliche Analyse nach malignen und benignen Hautläsionen zu untersuchen. Natürlich kann eine solche Anwendung aber keinen Mediziner ersetzen, sondern lediglich als zusätzliche Unterstützung dienen. 
\end{abstract}
    

\section{Einleitung}

Hautkrebs gilt als eine der häufigsten Krebserkrankungen der Welt. Jährlich erkranken etwa 18.000 Menschen in Deutschland an dieser Krankheit, wobei Hautkrebs allgemein etwa für ein Prozent der Krebstodesfälle verantwortlich ist \citep{hautkrebs}. Findet eine Erkennung der malignen Hautläsionen frühzeitig statt, so ist es in den meisten Fällen möglich einen tödlichen Verlauf der Krankheit zu verhindern. Daher sind frühzeitige Erkennungssysteme sehr wichtig für die Bekämpfung von Hautkrebs.\\
\noindent Eine Ergänzung zum regelmäßigen Arztbesuch und dem damit verbundenem Hautscreening, können daher neuronale Netze bieten, die aufgrund von medizinischen Datenbanken lernen können, eine maligne von einer benignen Hautläsion zu unterscheiden. Die genaue Implementierung und das Training solcher neuronalen Netzen werden wir im Folgenden genauer erklären. 
\section{Problemstellung und Zielsetzung}

Im Rahmen des Machine Learning Praktikums beschäftigen wir uns mit der folgenden Problemstellung: unser Ziel ist es, einen Klassifikator zu entwickeln, der Bilder von Hautläsionen in maligne und benigne Läsionen unterteilen kann. Maligne Hautläsionen sind die Läsionen, die für den Menschen gefährlich bis sogar tödlich werden könne, während benigne Hautläsionen die gutartigen Läsionen darstellen. Unser Projekt basiert dabei auf der Arbeit von \citet{esteva2017dermatologist}, wobei wir die originale Problemstellung jedoch etwas abgewandelt haben. Während \citet{esteva2017dermatologist} viele verschiedene Arten von Hautläsionen unterschieden haben, wollen wir lediglich zwischen zwei Klassen, nämlich den gutartigen und den bösartigen Läsionen, unterscheiden, was die Problemstellung etwas vereinfacht.\\
Unser Ziel war es eine möglichst hohe Genauigkeit zu erreichen und vor allem die Anzahl der falsch Negativen möglichst gering zu halten. Denn im Zweifel soll der Klassifikator eine Läsion als maligne klassifizieren, auch wenn sie eigentlich benigne ist, anstatt eine maligne Läsion, die tödlich verlaufen könnte, zu verharmlosen und als benigne zu klassifizieren. Im Folgenden werden wir genauer auf die Methodik eingehen, die hinter unserem Klassifikator steckt und welche Ergebnisse dieser auf unbekannte Bilder von Hautläsionen liefert. 
\section{Methoden und Tools}

Zur Erstellung des Klassifikators nutzten wir, wie \citet{esteva2017dermatologist} auch das öffentlich zur Verfügung stehende GoogleNet Inception v3, welches ein convolutional neural network ist und aus zahlreichen verschiedenen Schichten und Neuronen besteht \citep{szegedy2016rethinking} . Dieses neuronale Netz wurde auf den Bildern des ImageNet vortrainiert \citep{russakovsky2015imagenet} und eignet sich daher gut für unser Klassifizierungsproblem. \todo{Woher kommen die gewichte?} Das Training und die damit verbundene Vorverarbeitung programmierten wir mittels Python, in Kombination mit Tensorflow, Scikit Learn und NumPy. 

Als Datensatz nutzten wir die ISIC-Datenbank, welche aus insgesamt 13.768 Bildern von sowohl malignen als auch benignen Hautläsionen besteht. Von den Datensätzen, die von \citep{esteva2017dermatologist} genutzt wurden, ist dieser der Einzige, welcher frei verfügbar ist. Die Bilder dieses Datensatzes sind durch eine pathologische Untersuchung gelabelt worden. Somit sind die vorhandenen Daten relativ zuverlässig.

Es stellte sich heraus, dass die Bilder dieser Datenbank sehr unausgeglichen waren: der Anteil der malignen Läsionen war viel geringer als der Anteil der benignen Läsionen. Dieses Problem lösten wir durch eine spezielle Art der Randomisierung. Während des Trainings trennten wir die Menge der Trainingsbilder in die Klassen maligne und benigne auf. Anschließend haben wir die Minibatches, welche in das Netz gegeben wurden, aus Bildern dieser beiden Klassen zufällig befüllt. Dabei war die Wahrscheinlichkeit, dass ein Bild aus der Klasse der malignen Bilder stammt $p=0.5$. Somit wurde dieser unausgeglichene Datensatz ausbalanciert.

Ein weiteres Problem des ISIC-Datensatzes, war die variable Größe der einzelnen Bilder. Für ein problemloses Trainieren des Netzes, haben wir die Bilder daher vorverarbeitet. Dazu skalierten wir sie auf die kleinste verfügbare Größe. Für ads Training des Netzes verwendeten wir, analog zu \citep{esteva2017dermatologist}, das maximale zentrale Quadrat des Bildes und skalierten es auf 299x299 Pixel herunter. Somit hatten wir eine homogene Menge an Bildern, die wir nun in eine Trainings-, eine Validierungs- und eine Test-Menge unterteilten.

Während des Trainings wurden die Bilder zufällig augmentiert. Durch eine Augmentierung wird der Datensatz künstlich vergrößert um mehr diverse Trainingsbeispiele für das neuronale Netz zu erhalten. Dazu wendet man eine klassenerhaltende Transformation auf das Bild an. In unserem Ansatz verwenden wir verschiedene Augmentierungen, die jeweils zufällig auf ein Bild angewendet wurden. Wir verwendeten die folgenden Transformationen:

\begin{figure}[t!]
	\centering
	\begin{subfigure}{0.24\linewidth}
		\includegraphics[width=\textwidth]{./pics/augmentations/original.jpg}
		\caption{Original}
		\label{subfig:aug_original}
	\end{subfigure}
	\begin{subfigure}{0.24\linewidth}
		\includegraphics[width=\textwidth]{./pics/augmentations/brightness.png}
		\caption{Helligkeit}
		\label{subfig:aug_bright}
	\end{subfigure}
	\begin{subfigure}{0.24\linewidth}
		\includegraphics[width=\textwidth]{./pics/augmentations/contrast.png}
		\caption{Kontrast}
		\label{subfig:aug_contrast}
	\end{subfigure}
	\begin{subfigure}{0.24\linewidth}
		\includegraphics[width=\textwidth]{./pics/augmentations/hue.png}
		\caption{Hue}
		\label{subfig:aug_hue}
	\end{subfigure}
	\begin{subfigure}{0.24\linewidth}
		\includegraphics[width=\textwidth]{./pics/augmentations/rotation.png}
		\caption{Rotation}
		\label{subfig:aug_rot}
	\end{subfigure}
	\begin{subfigure}{0.24\linewidth}
		\includegraphics[width=\textwidth]{./pics/augmentations/vertical_flip.png}
		\caption{Vertikale Spiegelung}
		\label{subfig:aug_v_flip}
	\end{subfigure}
	\begin{subfigure}{0.24\linewidth}
		\includegraphics[width=\textwidth]{./pics/augmentations/horizontal_flip.png}
		\caption{Horizontale Spiegelung}
		\label{subfig:aug_h_flip}
	\end{subfigure}
	\begin{subfigure}{0.24\linewidth}
		\includegraphics[width=\textwidth]{./pics/augmentations/img_norm.png}
		\caption{Transformiert zu Imagenet Mean}
		 \todo{schöner schreiben}
		\label{subfig:aug_h_flip}
	\end{subfigure}
	\caption{Augmentierungsmethoden angewendet auf das orginale Bild (Abbildung \ref{subfig:aug_original})}
\end{figure}


\begin{enumerate}
    \item \textbf{Rotation:} Bei dieser Transformation werden die Bilder je um 0$^{\circ}$, 90$^{\circ}$, 180$^{\circ}$ oder 270$^{\circ}$ gedreht. Diese Augmentierung soll das Netz invariant gegenüber Rotation machen. Dies ist wichtig, da die Orientierung der aufgenommenen Bilder willkürlich ist.
    \item \textbf{Vertikales Spiegeln:} Hier werden die Bilder vertikal gespiegelt. Hierbei wird die Charakteristik der Läsion nicht verändert, jedoch werden so weitere Trainingsbilder erzeugt. 
    \item \textbf{Horizontales Spiegeln:} Hier werden die Bilder horizontal gespiegelt. Hierbei wird die Charakteristik der Läsion nicht verändert, jedoch werden so weitere Trainingsbilder erzeugt.
    \item \textbf{Helligkeit:} Die Helligkeit der Bilder wird hier um einen zufälligen \todo{welche range} Wert erhöht oder erniedrigt. So soll dem Netz eine gewisse Invarianz gegenüber verschiedenen Lichtbedingungen an trainiert werden.
    \item \textbf{Kontrast:} Analog zur Helligkeitsaugmentierung wird hier der Kontrast um einen zufälligen Wert erhöht oder erniedrigt. Auch dies soll das Netz invariant gegenüber wechselnden Lichtverhältnissen machen und zudem die Trainings-Menge vergrößern.
    \item \textbf{Hue:} Hier wird der Hue-Wert der Bilder zufällig geändert \todo{schöner schreiben!}. 
    \item \textbf{Sättigung:} Die Sättignung der Bilder wird zufällig höher oder niedriger gewählt. Auch dies soll gegen abweichende Lichtverhältnisse helfen und die Trianings-Menge vergrößern.
\end{enumerate}



\begin{itemize}
	\item Python, Tensorflow, Scikit Learn, NumPy
	\item Skalierung der Bilder
	\item Augmentierungsmethoden
	\item Trainingsparameter (Anzahl Durchläufe, Lernrate, Loss-Funktionen)
	\item Evaluationsmethoden: Berechnung des Scores, etc.
	\item Aufteilung des Datensatzes in Training, Test, Validierung
	\item genaue Erklärung des Shufflings, da Datensatz nicht ausbalanciert
\end{itemize}

\subsection{Analysemethoden}

Für die Bewertung und Optimierung unseres Klassifizierers wurden verschiedene Methoden angewandt. Die Genauigkeit konnte mit unserem Datensatz nur unter großer Skepsis betrachtet werden. Sie berechnet sich durch die Anzahl der richtig vorhergesagten Beispiele geteilt durch die Anzahl aller Beispiele:
	\[\text{Accuracy} = \frac{TP+TN}{TP+TN+FP+FN}\]
Da der in diesem Projekt verwendete Datensatz deutlich mehr negative als positive Beispiele enthält, können unter dieser Berechnung sehr hohe Genauigkeiten auftreten, obwohl eventuell kein positives Beispiel richtig berechnet wurde. Eine weitaus bessere Analyse ist hierbei möglich indem die richtig-positiv und richtig-negativen Beispiele getrennt angeschaut werden. Die Sensitivität, auch richtig-positiv Rate oder Recall genannt entspricht in unserem Fall den Anteil an tatsächlich malignen Hautläsionen, bei denen die Hautläsion auch als maligne erkannt wurde: 
\[\text{Sensitivität} = \frac{TP}{TP+FN}\]
Die Spezifität, auch richtig-negativ Rate genannt, entspricht dem Anteil an tatsächlich benignen Hautläsionen, die auch als benigne erkannt wurden:
\[\text{Spezifität} = \frac{TN}{TN+FP}\]
Die getrennte Betrachtung beider Werte erlaubt es uns direkt zu sehen ob maligne oder benigne Beispiele besser erkannt werden und dementsprechend zu optimieren.

Zusätzlich zogen wir außerdem noch den \textit{Matthews correlation coefficient} in Betracht. Dieser wird als Qualitätsmaßstab in binären maschinellen Lernmethoden verwendet, im Speziellen wenn der Datensatz sehr unausgeglichen ist. 

\[\text{MCC} = \frac{TP*TN - FP*FN}{\sqrt{(TP+FP)*(TP+FN)*(TN+FP)*(TN+FN)}}\]

Im Gegensatz zu den anderen hier verwendeten Methoden liefert der MCC Werte zwischen $-1$ und $1$. Liefert ein Klassifizierer einen MCC von $0$ so ist er nicht besser als der Zufall. Ein MCC von $1$ hingegen steht für eine komplette Übereinstimmung, ein MCC von $-1$ für die komplette Misklassifizierung.

Schließlich verwendeten wir noch den F1- und F2-Score als Qualitätsmaßstäbe. Der F1-Score ist das harmonische Mittel zwischen dem Recall und der Precision, der F2-Score gewichtet den Recall stärker und setzt somit den Schwerpunkt mehr auf die falsch negativen Stichproben, also die malignen Hautläsionen, die als benigne erkannt wurden:
	\[\text{Precision} = \frac{TP}{TP+FP}\]
    \[\text{Recall/Sensitivität} = \frac{TP}{TP+FN}\]
	\[\text{F1-Score} = 2*\frac{\text{precision}*\text{recall}}	{\text{precision}+\text{recall}}\]
   	\[\text{F2-Score} = 5*\frac{\text{precision}*\text{recall}}	{4*\text{precision}+\text{recall}}\]
    
Wir entschieden uns bewusst für eine größere Anzahl von Qualitätsmaßstäben. Da ein Parameter allein nie die Komplexität des Klassifizierers komplett beschreiben kann, sollen die verschiedenen Qualitätsmaßstäbe bei der Evaluierung und der Entscheidung für den für uns geeignetsten Klassifizierer als Gesamtheit betrachtet und verwendet werden. Dies ermöglicht es uns die Klassifizierung gezielt in die von uns gewünschte Richtung zu lenken um zum Beispiel Falsch-Negative Vorhersagen zu minimieren.

\section{Ergebnisse}
\color{red}
\begin{itemize}
	\item besten Trainingsdurchlauf genau beschreiben $\rightarrow$ Lernrate, Loss, ...
	\item Evaluierungswerte aus diesem Durchlauf nennen und beschreiben bzw. erklären
	\item viele Bildchen :-) 
	\item Bilder aus Tensorboard
	\item Matlab-Plots und Tabellen 
    \item Threshold, Table, Plots von bestem durchgang
\end{itemize}
\color{black}

Um einen guten Klassfizierer zu bekommen waren einige Trainingsdurchläufe notwendig. Zum testen der Klassifizierer wurden diese auf einen Validierungs-Datensatz angewandt. Die anfänglichen Klassifizierer haben meist nur minimal bessere Ergebnisse als der Zufall geliefert. Durch verschiedene Anpassungen die in Kapitel~\ref{training} näher erläutert wurden konnten schließlich besser Ergebnisse erreicht werden. Einige ausgewählte Ergebnisse werden in Abbildung~\ref{fig:roc} als ROC-Kurve dargestellt. Wie man gut erkennen kann verlaufen zwei der Kurven nahe der Diagonalen. Diese zwei Klassifikatoren (2018-03-12\_22-04-38 und 2018-03-12\_22-17-46) liefern keine guten Ergebnisse. Interessant und scheinbar gute Klassifikatoren verlaufen in der ROC-Kurve links oben und sind in diesem Fall 2018-03-10\_19-16-32, 2018-03-09\_21-34-4 und 2018-04-13\_20-38-20. Die anderen Klassifikatoren können als in Ordnung eingestuft werden.

\begin{figure}[htb!]
	\begin{center}
		\includegraphics[width=\textwidth]{./pics/evaluation/roc_analysis.png}
		\caption{ROC-Kurven einiger trainierter Klassifizierer, angewandt auf den Validierungs-Datensatz.}
		\label{fig:roc}
    \end{center}
\end{figure}

Um die ROC-Kurve nicht optisch zu evaluieren sondern mit einem Score zeigt Tabelle~\ref{tab:auc} die Klassifikatoren aus Abbildung~\ref{fig:roc} zusammen mit ihrer AUC. Wie man gut sehen kann gleichen sich die optischen Erkenntnisse mit den AUC-Werten. Somit ist der Klassifikator 2018-03-12\_22-04-38 mit einem AUC von 0.6 der Schlechteste und nur leicht besser als der Zufall und Klassifikator 2018-03-10\_19-16-32 mit einem AUC von 0.9 der Beste. 

\begin{table}[htb!]
\begin{center}
\begin{tabular}{ll}
	\toprule
 	Klassifikator  & AUC\\
	\midrule
  	2018-03-09\_21-34-47 &   $0.87$\\
    2018-03-10\_19-16-32 &   $0.90$\\
    2018-03-12\_22-04-38 &   $0.60$\\
    2018-03-12\_22-17-46 &   $0.65$\\
    2018-03-27\_14-12-20 &   $0.78$\\
    2018-04-13\_20-38-20 &   $0.86$\\
    2018-04-13\_20-41-14 &   $0.79$\\
    2018-04-13\_21-05-39 &   $0.75$\\
 \bottomrule
 \end{tabular}
 \end{center}
  \caption{ROC-AUC Ergebnisse einiger trainierten Klassifikatoren}
 \label{tab:auc}
 \end{table}
 
Da wir bei der Hautkrebs-Erkennung noch ein paar Sonderanforderungen an den Klassifikator haben, entschieden wir uns die besten Zwei noch genauer zu Evaluieren und nahmen somit zu unserem bisher besten Klassifikator 2018-03-10\_19-16-32 auch noch Klassifikator 2018-03-09\_21-34-47 mit einem AUC von $0.87$ hinzu zu nehmen. Unser Klassifizierer soll am Ende natürlich allgemein eine hohe Vorhersagegenauigkeit haben, allerdings sollten nicht zu viele wirklich positive Ergebnisse (maligne) als negativ (benigne) klassifiziert werden. 

Tabelle~\ref{tab:scores} zeigt die verschiedenen Qualitätsmaßstäbe,  die schon in Kapitel~\ref{analysemethoden} beschrieben wurden. Wie man der Tabelle gut entnehmen kann ist Klassifikator 2018-03-10\_19-16-32 nahezu allen belangen besser als 2018-03-09\_21-34-47. Einzig und allein in der Spezifität ist er minimal schlechter.

\begin{table}[htb!]
\begin{center}
\begin{tabular}{lllllllllll}
	\toprule
 	Klassifikator  & TP & FN & TN & FP & MCC & F2 & Acc & Sens & Spez\\
	\midrule
	2018-03-09\_21-34-47 & $119$ &	$115$ &	$2406$ &	$113$ &	$0.47$ &	$0.51$&	$0.92$ &	$0.51$ & $0.96$\\
    2018-03-10\_19-16-32 & $142$&	$91$ &	$2406$ &	$114$ &	$0.54$ 	&$0.60$	&$0.93$	&$0.61$&	$0.95$ \\
 \bottomrule
 \end{tabular}
 \end{center}
  \caption{Scores der zwei besten Klassifizierer: TP = True Positives, FN = False Negatives, TN = True Negatives, FP = False Positives, MCC = Matthews Correlation Coefficient, F2 = F2-Score, Acc = Genauigkeit, Sens = Sensitivität und Spez = Spezifität }
 \label{tab:scores}
 \end{table}
 
 Da unser Klassifikator aber eher maligne Hautläsionen auch wirklich als solche Erkennen sind uns präferieren wir in diesem Fall die wenigen Falsch negativen und die mehr wirklich positive Ergebnisse, wobei trotzdem nur ein negatives Ergebnis fälschlicherweise als positiv (maligne) eingeordnet wird. Aufgrund dieser Evaluierung schließen wir also das der Klassifizierer mit dem höherem AUC, MCC, F2-Score und Genauigkeit und Sensitivität unser Problem der Hautkrebserkennung besser lösen kann. Eine etwas geringere Spezifität würden wir hierbei in kauf nehmen, wobei diese in diesem geringen Maße auch eine Zufallserscheinung sein könnte.
 
Obwohl der Klassifizierer schon von einer recht guten Qualität ist werden die Ergebnisse durch die hohe Spezifität von 95\% dennoch etwas verfälscht. Mit einer Sensitivität von $0.6$ würden wir nämlich nur $60\%$ aller malignen Hautläsionen auch als wirklich maligne klassifizieren. Im Umkehrschluss heißt das vier von zehn potentielle Hautkrebs-Vorkommen würden unerkannt bleiben. Da diese Sensitivität noch zu nah an einer Zufallsklassifizierung liegt schauten wir uns die Ergebnisse des Neuronalen Netzes des  Klassifikators noch einmal genauer an. Das Ergebnis des Neuronalen Netz ist ein Score zwischen null und eins. Liegt der Score über $0.5$ wird eine Hautläsion als maligne klassifiziert, darunter als benigne. Da wir eine höhere Sensitivität erreichen wollten schoben wir also den \textit{threshold} für eine maligne Klassifizierung weiter in Richtung null. Abbildung~\ref{fig:threshold} zeigt wie sich eine Verschiebung des Thresholds auf die einzelnen Qualitätsmaßstäbe auswirkt. 

\begin{figure}[htb!]
	\begin{center}
		\includegraphics[width=\textwidth]{./pics/evaluation/treshold.png}
		\caption{Score-Threshold Abhängigkeit}
		\label{fig:threshold}
    \end{center}
\end{figure}

Wie man sehr gut erkennen kann gibt es einen Bereich zwischen $0.3$ und $0.6$ in der MCC und F2-Score gute Werte aufweisen. Da der F2-Score die falsch-negativen noch stärker bestraft konzentrierten wir uns hier noch einmal mehr auf diesen Qualitätsmaßstab. Auch eine hohe Sensitivität ist uns wichtig. Der F2-Score weist im Bereich $0.3$ und $0.45$ seine höchsten Werte auf, wobei der kleinste Threshold gleichzeitig die höchste Sensitivität ergibt. Es wäre also naheliegend einfach diesen Wert zu nehmen. Da die Spezifität zwischen dem Threshold $0.25$ und $0.3$ rapide ansteigt sollten ein gewisser Abstand zu diesem Bereich gewahrt werden. Somit entschieden wir uns den Threshold von den ursprünglichen $0.5$ auf $0.35$ herunterzusetzen. Damit halten wir genug Abstand von der extremen Änderung der Spezifität. Durch diese nun sensitivere Klassifizierung erhielten wir schließlich die Werte wie sie in Tabelle~\ref{final_scores} gelistet sind.

\begin{table}[htb!]
\begin{center}
\begin{tabular}{lllllllllll}
	\toprule
 	Threshold  & TP & FN & TN & FP & MCC & F2 & Acc & Sens & Spez\\
	\midrule
    $0.5$ & $142$&	$91$ &	$2406$ &	$114$ &	$0.54$ 	&$0.60$	&$0.93$	&$0.61$&	$0.95$ \\
	$0.35$ & $190$ &	$43$ &	$2255$ &	$265$ &	$0.55$ &	$0.68$&	$0.89$ &	$0.81$ & $0.89$\\
 \bottomrule
 \end{tabular}
 \end{center}
  \caption{Scores des Klassfizierers mit dem Threshold bei $0.35$}
 \label{tab:final_scores}
 \end{table}




\todo{Parameter für besten Trainingsdurchlauf}










\section{Aussicht}

\begin{itemize}
	\item Einbindung in App $\rightarrow$ unsere Fortschritte dort beschreiben
	\item Verknüpfung des Klassifikators mit speziellen Kameras fürs Handy
	\item Vorteil von ML in Medizin: Bewusstsein für Skin Cancer kann erhöht werden, einfachere Methode grad in ländlichen Gegenden (wo es nicht viele Ärzte gibt)
	\item genauere Unterscheidung wie auch im Paper noch möglich (dass einzelne Subtypen auch erkannt und unterschieden werden)

\end{itemize}

\bibliography{mylit}
\bibliographystyle{apalike}

\end{document}